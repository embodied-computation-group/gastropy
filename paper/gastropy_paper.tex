\documentclass[12pt]{article}

% --- Packages ---
\usepackage[utf8]{inputenc}
\usepackage[T1]{fontenc}
\usepackage{lmodern}
\usepackage[margin=1in]{geometry}
\usepackage{graphicx}
\usepackage{booktabs}
\usepackage{hyperref}
\usepackage{listings}
\usepackage{xcolor}
\usepackage{natbib}
\usepackage{amsmath}
\usepackage{setspace}
\usepackage{fancyhdr}

% --- Code listing style ---
\definecolor{codebg}{gray}{0.96}
\definecolor{codegreen}{rgb}{0.0,0.5,0.0}
\definecolor{codegray}{rgb}{0.5,0.5,0.5}
\definecolor{codepurple}{rgb}{0.58,0,0.82}

\lstdefinestyle{python}{
  language=Python,
  backgroundcolor=\color{codebg},
  basicstyle=\small\ttfamily,
  keywordstyle=\color{codepurple}\bfseries,
  stringstyle=\color{codegreen},
  commentstyle=\color{codegray}\itshape,
  breaklines=true,
  frame=single,
  framerule=0pt,
  xleftmargin=1em,
  xrightmargin=1em,
  aboveskip=1em,
  belowskip=1em,
  showstringspaces=false,
  columns=fullflexible,
  keepspaces=true,
  morekeywords={as, with, True, False, None},
}

\lstset{style=python}

% --- Metadata ---
\title{GastroPy: An Open-Source Python Toolbox for Electrogastrography
  Signal Processing and Gastric-Brain Coupling Analysis}

\author{
  Micah Allen\textsuperscript{1,2,*},
  Daniel Kluger\textsuperscript{3},
  Leah Banellis\textsuperscript{1,2},
  Ignacio Rebollo\textsuperscript{4},\\
  Nils Kroemer\textsuperscript{5},
  Edwin Dalmaijer\textsuperscript{6}\\[1em]
  \small\textsuperscript{1}Center of Functionally Integrative Neuroscience,
    Aarhus University, Denmark\\
  \small\textsuperscript{2}Cambridge Psychiatry,
    University of Cambridge, United Kingdom\\
  \small\textsuperscript{3}Institute of Biomagnetism and Biosignal Analysis,
    University of Muenster, Germany\\
  \small\textsuperscript{4}German Institute for Human Nutrition,
    Potsdam, Germany\\
  \small\textsuperscript{5}Department of Psychiatry and Psychotherapy,
    University of Tuebingen, Germany\\
  \small\textsuperscript{6}School of Psychology,
    University of Leeds, United Kingdom\\[0.5em]
  \small\textsuperscript{*}Corresponding author: \texttt{micah@cfin.au.dk}
}

\date{}

\begin{document}

\maketitle

% ============================================================================
% ABSTRACT
% ============================================================================
\begin{abstract}
Electrogastrography (EGG) is a non-invasive technique for recording gastric
myoelectrical activity via cutaneous abdominal electrodes. Recent advances in
concurrent EGG-fMRI have revealed phase synchrony between the gastric rhythm
and resting-state brain networks, opening a new window into brain-body
interactions, interoception, and mental health. Despite growing interest, the
field lacks a dedicated, open-source software toolkit for EGG analysis in
Python. Here we present GastroPy, a modular Python package for EGG signal
processing and gastric-brain coupling analysis. GastroPy provides a complete
analysis pipeline spanning spectral analysis, bandpass filtering,
instantaneous phase extraction, cycle-level metrics, multi-channel selection,
phase-based artifact detection, time-frequency decomposition, and
phase-locking value (PLV) computation. A dedicated neuroimaging sub-module
supports scanner trigger alignment, confound regression, voxelwise BOLD phase
extraction, and whole-brain PLV map generation for concurrent EGG-fMRI data.
The package is built on NumPy and SciPy with optional MNE-Python and nilearn
\citep{Abraham2014} integration, features a layered architecture that separates core signal
processing from neuroimaging workflows, and includes bundled sample data for
tutorials and testing. We describe GastroPy's design philosophy, demonstrate
its capabilities through two worked examples (standalone EGG processing and
fMRI-EGG coupling), and discuss its role in the emerging ecosystem of
brain-body analysis tools. GastroPy is freely available under the MIT license
at \url{https://github.com/embodied-computation-group/gastropy}.
\end{abstract}

\onehalfspacing

% ============================================================================
% 1. INTRODUCTION
% ============================================================================
\section{Introduction}

The stomach generates a continuous electrical slow wave at approximately 0.05
Hz (3 cycles per minute) that originates in the interstitial cells of Cajal
(ICC) located in the myenteric plexus of the gastric corpus, and propagates
distally through the antrum to coordinate gastric motility
\citep{Koch2004, ChenMcCallum1991}. This omnipresent rhythm---present in both
fasting and postprandial states---serves as the pacemaker signal that triggers
smooth muscle contraction and governs the mixing and propulsion of gastric
contents. The gastric slow wave can be recorded non-invasively using
electrogastrography (EGG), a technique in which cutaneous electrodes placed on
the abdominal skin surface capture the far-field projection of the underlying
myoelectrical activity \citep{Chang2005, Yin2013}. The resulting signal,
though weak in amplitude and susceptible to respiratory, cardiac, and motion
artifacts, provides a continuous measure of gastric rhythmicity that can be
characterized through spectral analysis, phase extraction, and cycle-level
metrics.

Clinical EGG has been used for decades to assess gastric dysrhythmias
(bradygastria, tachygastria), gastroparesis, functional dyspepsia, and other
motility disorders \citep{Koch2004}. Standard clinical parameters include the
dominant frequency, percentage of recording time in the normal 2--4 cycles per
minute (cpm) range, the power ratio between pre- and postprandial recordings,
and the instability coefficient quantifying rhythm regularity. However, the
technique has recently attracted renewed interest from the neuroscience and
psychophysiology communities, driven by discoveries of gastric-brain
interactions that may have implications far beyond gastroenterology.

A landmark study by \citet{Rebollo2018} demonstrated that the gastric rhythm
is phase-synchronized with resting-state fMRI activity in a distributed
``gastric network'' spanning somatosensory, motor, visual, and parietal
cortices. Using concurrent EGG and fMRI in healthy participants, they showed
that the phase of the gastric slow wave predicts fluctuations in the BOLD
signal at specific brain regions, with a precise temporal sequence of
activations within each gastric cycle. This finding was subsequently replicated
and extended by \citet{Choe2021}, who used spatial independent component
analysis (ICA) to show that at least three resting-state networks---cerebellar,
dorsal somatosensory-motor, and default mode---exhibit significant
phase-locking with the gastric basal electrical rhythm. \citet{Levakov2023}
further characterized the reliability and confound sensitivity of
gastric-brain coupling measures, demonstrating that head motion and
non-grey-matter signals can substantially inflate the spatial extent of the
gastric network if not properly controlled. Most recently,
\citet{Banellis2025} demonstrated that frontoparietal brain coupling to the
gastric rhythm indexes a dimensional signature of mental health, with
cross-validated machine learning revealing that increased gastric-brain
coupling in frontoparietal regions predicts poorer anxiety, depression, stress,
and well-being across 243 participants.

Alongside these neuroscience applications, \citet{Wolpert2020} established
practical recording and analysis guidelines for EGG in psychophysiology
research, providing normative data from a large healthy sample (N = 117) and a
semi-automated MATLAB analysis pipeline that has become a de facto standard for
EGG data processing. Their pipeline includes electrode placement
recommendations, signal quality assessment criteria, and a multi-step analysis
workflow from raw recording to phase and power extraction. Wearable EGG devices
have also been explored for affect detection and emotion regulation
applications \citep{Vujic2020}, demonstrating that gastric signals carry
affective information complementary to other autonomic measures and further
expanding the potential user base for EGG analysis tools.

Despite this growth, the field currently lacks a dedicated, open-source Python
package for EGG analysis. Researchers typically rely on custom MATLAB scripts,
ad hoc Python code, or general-purpose biosignal toolkits that were not
designed for the unique characteristics of the gastric slow wave. NeuroKit2
\citep{Makowski2021} provides broad biosignal processing capabilities with
basic EGG support but lacks multi-channel selection, gastric-brain coupling
pipelines, and EGG-specific artifact detection. MNE-Python
\citep{Gramfort2013} is the de facto standard for EEG and MEG analysis but
does not target the low-frequency range occupied by the gastric rhythm. Domain-specific
biosignal packages such as Systole \citep{Legrand2020} and HeartPy
\citep{vanGent2019} demonstrate the value of focused toolkits but address
cardiac rather than gastric signals. MATLAB toolboxes such as the StomachBrain
pipeline \citep{Banellis2025} are not accessible to the growing number of
researchers working in Python, and their monolithic architectures limit
component reusability.

Here we present GastroPy, an open-source Python package that provides a
complete, modular toolkit for EGG signal processing and gastric-brain coupling
analysis. GastroPy is designed around three principles: (1) domain specificity,
with frequency bands, filtering parameters, and artifact criteria tailored to
the gastric slow wave; (2) modularity, with a layered architecture that
separates core DSP from neuroimaging-specific logic; and (3) interoperability,
with all functions operating on standard NumPy arrays and pandas DataFrames. In
what follows, we describe the package's design philosophy and architecture,
demonstrate its capabilities through two worked examples, and discuss its role
in the emerging ecosystem of brain-body analysis tools.

% ============================================================================
% 2. DESIGN PHILOSOPHY
% ============================================================================
\section{Design Philosophy and Architecture}

\subsection{Guiding Principles}

GastroPy's design is guided by four principles drawn from best practices in
scientific Python software development, particularly the approaches pioneered
by NeuroKit2 \citep{Makowski2021} and MNE-Python \citep{Gramfort2013}:

\begin{enumerate}
  \item \textbf{Domain-specific defaults with full parameter control.} All
    functions expose sensible defaults for gastric rhythm analysis (e.g.,
    normogastric band of 0.033--0.067 Hz, 200-second PSD windows) while
    allowing complete parameter customization for advanced users.
  \item \textbf{Layered dependency isolation.} Core signal processing depends
    only on NumPy \citep{Harris2020}, SciPy \citep{Virtanen2020}, pandas
    \citep{McKinney2010}, and matplotlib \citep{Hunter2007}. Neuroimaging
    features (MNE-Python, nilearn, nibabel) are optional extras, keeping the
    base install lightweight.
  \item \textbf{Composable, flat API.} Each function operates on NumPy arrays
    or pandas DataFrames and can be used independently or composed into
    pipelines. High-level convenience functions aggregate lower-level
    operations but never hide them.
  \item \textbf{Reproducibility by default.} All random operations (e.g.,
    surrogate generation) accept explicit seeds. Bundled sample data and
    comprehensive documentation ensure that analyses can be reproduced across
    labs and platforms.
\end{enumerate}

\subsection{Package Architecture}

GastroPy is organized into seven modules arranged in two layers
(Table~\ref{tab:modules}).

\begin{table}[htbp]
\centering
\caption{GastroPy module organization. Core modules require only NumPy, SciPy,
  pandas, and matplotlib. The neuroimaging layer requires optional dependencies
  (MNE-Python, nilearn, nibabel), installable via
  \texttt{pip install gastropy[neuro]}.}
\label{tab:modules}
\begin{tabular}{lll}
\toprule
\textbf{Layer} & \textbf{Module} & \textbf{Description} \\
\midrule
Core & \texttt{gastropy.signal} & PSD, filtering, phase extraction, resampling \\
     & \texttt{gastropy.metrics} & Band power, instability, cycle statistics \\
     & \texttt{gastropy.egg} & High-level pipeline, channel selection \\
     & \texttt{gastropy.timefreq} & Narrowband decomposition, Morlet wavelets \\
     & \texttt{gastropy.coupling} & PLV, circular statistics, surrogates \\
     & \texttt{gastropy.viz} & Publication-ready plotting \\
     & \texttt{gastropy.data} & Bundled sample datasets \\
\midrule
Neuro & \texttt{gastropy.neuro.fmri} & Trigger alignment, BOLD phases, PLV maps \\
\bottomrule
\end{tabular}
\end{table}

The \textbf{core layer} handles all EGG-specific signal processing.
\texttt{gastropy.signal} provides low-level DSP primitives: power spectral
density via Welch's method, FIR and IIR bandpass filter design and application,
Hilbert-based instantaneous phase extraction, signal resampling, cycle edge
detection, and phase-based artifact detection. \texttt{gastropy.metrics}
computes gastric rhythm metrics including band power (peak frequency, maximum
power, proportional power), the instability coefficient
\citep{Koch2004}, cycle duration statistics, proportion of normogastric
cycles, and automated quality assessment. \texttt{gastropy.egg} composes these
into high-level workflows: \texttt{egg\_process} runs a complete
filter-phase-metrics pipeline in a single call, while
\texttt{select\_best\_channel} identifies the channel with the strongest
normogastric rhythm in multi-electrode recordings.
\texttt{gastropy.timefreq} provides narrowband decomposition via per-band
filtering and Morlet wavelet time-frequency representations.
\texttt{gastropy.coupling} implements modality-agnostic circular statistics
(circular mean, resultant length, Rayleigh test) and phase-locking value
computation with surrogate-based z-scoring via circular time-shifting.
\texttt{gastropy.viz} provides plotting functions for every analysis stage,
from PSD plots with gastric band shading to comprehensive multi-panel overview
figures.

The \textbf{neuroimaging layer} extends the core with fMRI-specific
functionality. \texttt{gastropy.neuro.fmri} provides scanner trigger
detection, volume window creation, per-volume phase extraction, transient
volume removal, confound regression via GLM, voxelwise BOLD phase extraction
at the gastric frequency, and whole-brain PLV map generation with
surrogate-based null distributions. This module implements the methodology
described in \citet{Banellis2025}.

\subsection{Gastric Frequency Bands}

GastroPy defines three standard gastric frequency bands as named constants in
\texttt{gastropy.metrics}:

\begin{itemize}
  \item \textbf{Bradygastria}: 0.02--0.03 Hz (1.2--1.8 cycles per minute)
  \item \textbf{Normogastria}: 0.033--0.067 Hz (2--4 cycles per minute)
  \item \textbf{Tachygastria}: 0.07--0.17 Hz (4.2--10.2 cycles per minute)
\end{itemize}

These bands follow established clinical conventions
\citep{Koch2004, Chang2005, Yin2013} and serve as defaults throughout the
package while remaining fully configurable via the \texttt{GastricBand}
named tuple.

% ============================================================================
% 3. KEY ALGORITHMS
% ============================================================================
\section{Key Algorithms and Methods}

This section describes the core signal processing and statistical methods
implemented in GastroPy.

\subsection{Spectral Analysis}

GastroPy computes power spectral density using Welch's method with a default
window length of 200 seconds, chosen to provide sufficient frequency
resolution in the gastric range (0.005 Hz resolution) while allowing multiple
windows for variance reduction in typical recording durations (15--30 minutes).
Users can control the overlap fraction (default 25\%) via the
\texttt{overlap} parameter. The resulting PSD is used for channel selection,
peak frequency identification, and band power computation.

Band power metrics are computed by integrating the PSD within each gastric
band. For each band, GastroPy reports the peak frequency (Hz), maximum and
mean power, proportional power (band power divided by total power in the
0.01--0.2 Hz analysis range), and the mean power ratio (band power relative to
out-of-band power). These metrics follow the conventions established in the
clinical EGG literature \citep{Koch2004, Chang2005}.

\subsection{Bandpass Filtering}

GastroPy supports both FIR and IIR bandpass filtering with zero-phase
application via \texttt{scipy.signal.filtfilt}. The default FIR filter uses an
adaptive tap count computed from the filter order, sampling frequency, and a
configurable transition width (default 20\% of the passband). IIR filtering
uses a Butterworth design with configurable order (default 4). The zero-phase
application ensures no phase distortion, which is critical for subsequent
Hilbert-based phase extraction.

For the default normogastric band (0.033--0.067 Hz), the filter passband
corresponds to cycle durations of 15--30 seconds. The transition width
parameter controls the sharpness of the roll-off and can be adjusted to balance
between frequency selectivity and temporal ringing artifacts.

\subsection{Phase Extraction and Cycle Detection}

Instantaneous phase is extracted via the Hilbert transform applied to the
bandpass-filtered signal. The analytic signal $z(t) = x(t) + i\hat{x}(t)$,
where $\hat{x}(t)$ is the Hilbert transform of $x(t)$, yields the
instantaneous phase as $\phi(t) = \arg(z(t))$ and the instantaneous amplitude
as $A(t) = |z(t)|$. The phase is wrapped to $[-\pi, \pi)$.

Gastric cycles are detected by identifying phase wrapping events (transitions
from $+\pi$ to $-\pi$). The duration of each cycle is computed as the time
between consecutive wrapping events. Cycle-level metrics include the mean and
standard deviation of cycle durations, the proportion of cycles falling within
the normogastric range (15--30 seconds by default), and the instability
coefficient (IC), defined as the standard deviation of cycle durations divided
by the mean:
\begin{equation}
  \text{IC} = \frac{\sigma_{\text{cycle}}}{\mu_{\text{cycle}}}
\end{equation}
Lower IC values indicate a more regular gastric rhythm. GastroPy also provides
automated quality assessment based on configurable thresholds for minimum cycle
count, rhythm stability, and normogastric dominance.

\subsection{Artifact Detection}

GastroPy implements phase-based artifact detection following the approach of
\citet{Wolpert2020}. Two types of artifacts are identified at the cycle level:
(1) \emph{non-monotonic phase} cycles, where the instantaneous phase does not
progress monotonically within a cycle (indicating loss of a stable oscillation),
and (2) \emph{duration outlier} cycles, where the cycle duration exceeds a
configurable threshold (default: 3 standard deviations from the mean).
Artifact cycles are flagged with a boolean mask that can be used to exclude
contaminated data from subsequent analyses.

\subsection{Phase-Locking Value}

The phase-locking value (PLV) quantifies the consistency of the phase
difference between two signals across time \citep{Rebollo2018}. For two phase
time series $\phi_a(t)$ and $\phi_b(t)$, the PLV is defined as:
\begin{equation}
  \text{PLV} = \left| \frac{1}{N} \sum_{t=1}^{N} e^{i(\phi_a(t) - \phi_b(t))} \right|
\end{equation}
where $N$ is the number of time points. PLV ranges from 0 (no phase
consistency) to 1 (perfect phase-locking). GastroPy also computes the complex
PLV, which preserves the preferred phase lag direction.

Statistical significance of observed PLV values is assessed using a
surrogate-based approach. Surrogate distributions are generated by circularly
shifting one phase time series by random offsets (drawn uniformly from the
full signal length, with a configurable buffer to avoid trivially similar
shifts). The empirical PLV is then z-scored against the surrogate distribution:
\begin{equation}
  z = \frac{\text{PLV}_{\text{emp}} - \text{median}(\text{PLV}_{\text{surr}})}
           {\text{MAD}(\text{PLV}_{\text{surr}})}
\end{equation}
where MAD is the median absolute deviation. This non-parametric approach avoids
distributional assumptions and is robust to non-stationarity in the signals.

\subsection{Voxelwise BOLD Phase Extraction}

For fMRI coupling analysis, GastroPy extracts the instantaneous phase of BOLD
signal fluctuations at the gastric frequency for each brain voxel. The BOLD
time series (after confound regression and z-scoring) is bandpass-filtered at
the participant's peak gastric frequency using either an IIR Butterworth filter
(default) or a FIR filter, with a configurable half-width at half-maximum
(HWHM, default 0.015 Hz) defining the filter bandwidth. The Hilbert transform
is then applied to extract the instantaneous phase. Transient volumes at the
beginning and end of the filtered time series are removed (default: 21 volumes
each) to avoid edge artifacts from the filtering and Hilbert transform
operations.

% ============================================================================
% 4. INSTALLATION
% ============================================================================
\section{Installation and Dependencies}

GastroPy is available on the Python Package Index (PyPI) and can be installed
with pip:

\begin{lstlisting}
pip install gastropy
\end{lstlisting}

This installs the core package with its minimal dependencies: NumPy, SciPy,
pandas, and matplotlib. For neuroimaging workflows that require MNE-Python,
nilearn, and nibabel:

\begin{lstlisting}
pip install gastropy[neuro]
\end{lstlisting}

For development, including testing and documentation tools:

\begin{lstlisting}
pip install gastropy[dev]
\end{lstlisting}

GastroPy requires Python 3.10 or later and is tested on Linux, macOS, and
Windows via continuous integration. The package uses Hatch as its build system,
Ruff for linting and formatting, and Sphinx with the sphinx-book-theme for
documentation, which is hosted at
\url{https://embodied-computation-group.github.io/gastropy}.

% ============================================================================
% 4. EXAMPLE 1: STANDALONE EGG
% ============================================================================
\section{Example 1: Standalone EGG Processing}

This example demonstrates a complete EGG analysis workflow using a standalone
recording from the \citet{Wolpert2020} normative dataset, bundled with
GastroPy.

\subsection{Loading Data}

GastroPy includes several sample datasets accessible via the
\texttt{gastropy.data} module. The standalone EGG recording contains 7 channels
sampled at 10 Hz from a healthy participant:

\begin{lstlisting}
import gastropy as gp

# Load standalone EGG recording (Wolpert et al., 2020)
rec = gp.load_egg()
signal = rec["signal"]   # shape: (n_channels, n_samples)
sfreq = rec["sfreq"]     # 10.0 Hz

print(f"Channels: {rec['ch_names']}")
print(f"Duration: {rec['duration_s']:.0f} s")
\end{lstlisting}

\subsection{Channel Selection}

For multi-channel recordings, GastroPy identifies the channel with the
strongest normogastric peak using spectral analysis:

\begin{lstlisting}
best_idx, peak_freq, freqs, psd = gp.select_best_channel(
    signal, sfreq
)
print(f"Best channel: {rec['ch_names'][best_idx]}")
print(f"Peak frequency: {peak_freq:.4f} Hz "
      f"({peak_freq * 60:.1f} cpm)")
\end{lstlisting}

The channel selection algorithm computes the PSD for each channel using
Welch's method, identifies spectral peaks within the normogastric band, and
selects the channel whose peak has the highest power. The PSD can be
visualized with gastric band shading:

\begin{lstlisting}
fig, ax = gp.plot_psd(
    freqs, psd,
    band=gp.NORMOGASTRIA,
    ch_names=rec["ch_names"],
    best_idx=best_idx,
    peak_freq=peak_freq,
)
\end{lstlisting}

\subsection{Signal Processing Pipeline}

The \texttt{egg\_process} function runs the full analysis pipeline on the
selected channel: bandpass filtering in the normogastric band, Hilbert-based
instantaneous phase and amplitude extraction, cycle detection, and metric
computation:

\begin{lstlisting}
signals_df, info = gp.egg_process(
    signal[best_idx], sfreq
)

# signals_df columns: raw, filtered, phase, amplitude
# info dict contains metrics:
print(f"Cycles detected: {info['n_cycles']}")
print(f"Mean cycle duration: "
      f"{info['mean_cycle_dur_s']:.1f} s")
print(f"Instability coefficient: "
      f"{info['instability_coefficient']:.3f}")
print(f"Proportion normogastric: "
      f"{info['proportion_normogastric']:.1%}")
\end{lstlisting}

The returned \texttt{signals\_df} DataFrame contains aligned time series of the
raw signal, bandpass-filtered signal, instantaneous phase (in radians), and
analytic amplitude. The \texttt{info} dictionary contains all computed metrics,
including the instability coefficient \citep{Koch2004}, cycle duration
statistics, band power, and an automated quality assessment flag.

\subsection{Artifact Detection}

Phase-based artifact detection, following the approach of
\citet{Wolpert2020}, identifies cycles with non-monotonic phase progression or
anomalous durations:

\begin{lstlisting}
times = signals_df.index.values / sfreq
artifact_info = gp.detect_phase_artifacts(
    signals_df["phase"].values, times
)

print(f"Artifact cycles: "
      f"{artifact_info['n_artifact_cycles']}")

fig, ax = gp.plot_artifacts(
    signals_df["phase"].values,
    times,
    artifact_info,
)
\end{lstlisting}

\subsection{Visualization}

GastroPy provides several plotting functions for visualizing results. The
\texttt{plot\_egg\_overview} function generates a four-panel figure showing the
raw signal, filtered signal, instantaneous phase, and amplitude envelope:

\begin{lstlisting}
fig, axes = gp.plot_egg_overview(signals_df, sfreq)
\end{lstlisting}

Cycle duration distributions can be visualized as histograms with the
normogastric range highlighted:

\begin{lstlisting}
fig, ax = gp.plot_cycle_histogram(
    info["cycle_durations_s"]
)
\end{lstlisting}

\subsection{Time-Frequency Analysis}

For more detailed spectral dynamics, the \texttt{multiband\_analysis} function
decomposes the signal across all three gastric bands simultaneously:

\begin{lstlisting}
bands = gp.multiband_analysis(signal[best_idx], sfreq)

for name, result in bands.items():
    bp = result["band_power"]
    print(f"{name}: peak {bp['peak_freq_hz']:.4f} Hz, "
          f"power {bp['prop_power']:.1%}")
\end{lstlisting}

% ============================================================================
% 5. EXAMPLE 2: fMRI COUPLING
% ============================================================================
\section{Example 2: Gastric-Brain Coupling with fMRI}

This example demonstrates the gastric-brain coupling pipeline using concurrent
EGG-fMRI data, implementing the methodology described in
\citet{Banellis2025}. The workflow proceeds from EGG phase extraction through
voxelwise BOLD phase computation to whole-brain PLV mapping.

\subsection{Loading Concurrent EGG-fMRI Data}

GastroPy bundles three sessions of concurrent EGG-fMRI recordings, each
containing 8-channel EGG at 10 Hz with scanner trigger timing:

\begin{lstlisting}
import gastropy as gp
from gastropy.neuro import fmri
import numpy as np

# Load EGG recording with trigger info
rec = gp.load_fmri_egg(session="0001")
signal = rec["signal"]   # (8, n_samples)
sfreq = rec["sfreq"]     # 10.0 Hz
tr = rec["tr"]            # 1.856 s

# Select best channel and get peak frequency
best_idx, peak_freq, freqs, psd = (
    gp.select_best_channel(signal, sfreq)
)
\end{lstlisting}

\subsection{EGG Phase Extraction}

The EGG signal is bandpass-filtered and the instantaneous phase is extracted
via the Hilbert transform:

\begin{lstlisting}
# Process EGG on best channel
signals_df, info = gp.egg_process(
    signal[best_idx], sfreq
)
egg_phase = signals_df["phase"].values
\end{lstlisting}

\subsection{Volume Windowing and Phase-per-Volume}

Scanner trigger times are used to create volume windows that map each fMRI
volume to the corresponding EGG time points. The mean EGG phase within each
volume window is then computed:

\begin{lstlisting}
# Create volume windows from trigger onsets
trigger_times = rec["trigger_times"]
n_volumes = len(trigger_times)
windows = fmri.create_volume_windows(
    trigger_times, tr, n_volumes
)

# Extract analytic signal for phase computation
_, analytic = gp.instantaneous_phase(
    gp.apply_bandpass(signal[best_idx], sfreq,
                      gp.NORMOGASTRIA.low,
                      gp.NORMOGASTRIA.high)
)

# Compute mean phase per fMRI volume
phase_per_vol = fmri.phase_per_volume(
    analytic, windows
)
\end{lstlisting}

\subsection{BOLD Preprocessing and Phase Extraction}

Preprocessed BOLD data (e.g., from fMRIPrep) is loaded, confound signals are
regressed out, and the instantaneous phase at the gastric frequency is
extracted for each voxel:

\begin{lstlisting}
# Load preprocessed BOLD data
bold_data = gp.fetch_fmri_bold(session="0001")
bold_2d = bold_data["bold"]       # (n_voxels, n_vols)
confounds = bold_data["confounds"] # DataFrame

# Regress out motion and physiological confounds
bold_clean = fmri.regress_confounds(bold_2d, confounds)

# Apply volume cuts to remove edge transients
begin_cut, end_cut = 21, 21
bold_cut = fmri.apply_volume_cuts(
    bold_clean, begin_cut, end_cut
)
egg_cut = fmri.apply_volume_cuts(
    phase_per_vol.reshape(1, -1),
    begin_cut, end_cut
).ravel()

# Extract BOLD phase at gastric frequency
bold_sfreq = 1.0 / tr
bold_phases = fmri.bold_voxelwise_phases(
    bold_cut, peak_freq, bold_sfreq
)
\end{lstlisting}

\subsection{PLV Map Computation}

The phase-locking value between the EGG phase and each voxel's BOLD phase is
computed to generate a whole-brain PLV map. Statistical significance is
assessed via comparison to a surrogate distribution created by circular
time-shifting:

\begin{lstlisting}
# Compute empirical PLV map
plv_map = fmri.compute_plv_map(egg_cut, bold_phases)

# Compute surrogate PLV for statistical testing
surr_plv = fmri.compute_surrogate_plv_map(
    egg_cut, bold_phases,
    n_surrogates=1000, seed=42
)

# Z-score the empirical PLV against surrogates
z_map = gp.coupling_zscore(plv_map, surr_plv)
print(f"Max PLV: {plv_map.max():.3f}")
print(f"Max z-score: {z_map.max():.2f}")
\end{lstlisting}

The resulting PLV and z-score maps can be reshaped to 3D volumes and saved as
NIfTI images for visualization in standard neuroimaging viewers, or plotted
directly using GastroPy's nilearn-based visualization functions.

\subsection{Coupling Statistics}

GastroPy's coupling module also provides functions for testing the
significance of individual PLV values using the Rayleigh test for circular
uniformity:

\begin{lstlisting}
# Test whether EGG-BOLD coupling is significant
# for a region of interest
roi_phase = bold_phases[roi_mask].mean(axis=0)
phase_diff = egg_cut - roi_phase

z_stat, p_value = gp.rayleigh_test(phase_diff)
print(f"Rayleigh z = {z_stat:.2f}, p = {p_value:.4f}")
\end{lstlisting}

% ============================================================================
% 6. COMPARISON WITH EXISTING TOOLS
% ============================================================================
\section{Comparison with Existing Tools}

Table~\ref{tab:comparison} compares GastroPy's capabilities with existing EGG
analysis tools.

\begin{table}[htbp]
\centering
\caption{Comparison of EGG analysis tools across key features. PLV = phase-locking value.}
\label{tab:comparison}
\begin{tabular}{lccccc}
\toprule
\textbf{Feature} & \textbf{GastroPy} & \textbf{NeuroKit2} & \textbf{MNE-Python} & \textbf{StomachBrain} \\
\midrule
Language & Python & Python & Python & MATLAB \\
EGG pipeline & Full & Basic & No & Partial \\
Multi-channel selection & Yes & No & N/A & No \\
Phase artifact detection & Yes & No & EEG-focused & No \\
Gastric-brain coupling & PLV + surr. & No & No & PLV \\
Time-frequency analysis & Yes & Yes & Yes & No \\
Bundled EGG data & Yes & No & No & No \\
Quality assessment & Automated & No & N/A & Manual \\
\bottomrule
\end{tabular}
\end{table}

GastroPy addresses several gaps not filled by existing tools. Unlike
NeuroKit2 \citep{Makowski2021}, which provides general biosignal processing
with basic EGG support (single-channel filtering and PSD), GastroPy offers a
complete EGG-specific pipeline including multi-channel selection, phase-based
artifact detection following \citet{Wolpert2020}, automated quality
assessment, and the full gastric-brain coupling workflow. Unlike MNE-Python
\citep{Gramfort2013}, which excels at high-frequency electrophysiology (EEG,
MEG) but does not target the gastric frequency range, GastroPy's filtering
and spectral analysis are optimized for the ultra-low-frequency (0.01--0.2
Hz) domain of gastric signals. Unlike the MATLAB-based StomachBrain pipeline
\citep{Banellis2025}, GastroPy provides a modular, pip-installable Python
package with a composable API, bundled test data, and continuous integration.

We chose to develop a standalone package rather than contribute EGG-specific
functionality to an existing project because the gastric slow wave has unique
characteristics that require domain-specific design decisions throughout the
analysis chain. The gastric rhythm occupies a frequency range (0.03--0.07 Hz)
that falls below the default filter settings of most electrophysiology
toolkits. Its long cycle duration (15--30 seconds) requires specialized
windowing parameters for spectral analysis, and the coupling with brain
signals requires fMRI-specific preprocessing (trigger alignment, confound
regression) that is outside the scope of general biosignal packages. GastroPy's
layered architecture makes it straightforward to use the core EGG processing
modules independently of the neuroimaging layer, and its standard array-based
API ensures interoperability with the broader scientific Python ecosystem.

% ============================================================================
% 7. TESTING AND QUALITY ASSURANCE
% ============================================================================
\section{Testing and Quality Assurance}

GastroPy maintains a comprehensive automated test suite of 178 tests covering
all public functions across every module. Tests verify signal processing
correctness (e.g., that bandpass filtering preserves in-band energy while
attenuating out-of-band frequencies), metric computation accuracy (e.g., that
the instability coefficient increases for irregular rhythms), coupling
statistics (e.g., that PLV equals 1.0 for identical phase series and
approaches 0 for uncorrelated series), and visualization outputs (e.g., that
plotting functions return valid matplotlib figures without errors).

The test suite runs automatically via GitHub Actions on every push and pull
request, with separate workflows for testing (pytest), linting (Ruff check and
format verification), and documentation building (Sphinx). Tests are executed
on the latest Ubuntu runner with Python 3.13. All sample data used in tests is
bundled with the package, ensuring that tests are self-contained and
reproducible without network access.

% ============================================================================
% 8. DISCUSSION
% ============================================================================
\section{Discussion}

GastroPy provides the first dedicated, open-source Python package for
electrogastrography signal processing and gastric-brain coupling analysis. By
consolidating methods that were previously scattered across lab-specific
MATLAB scripts and ad hoc Python code into a tested, documented, and
pip-installable package, GastroPy aims to lower the barrier to rigorous EGG
analysis and accelerate research on the brain-gut axis.

\subsection{Current Capabilities and Use Cases}

The package currently supports the full standalone EGG processing workflow
(spectral analysis, filtering, phase extraction, cycle metrics, artifact
detection, quality assessment) and the complete EGG-fMRI coupling pipeline
(trigger alignment, BOLD preprocessing, voxelwise phase extraction, PLV
mapping with surrogate testing). The modular architecture allows each
function to be used independently, enabling researchers to build custom
pipelines tailored to their specific experimental designs while benefiting
from validated, tested implementations of standard analysis steps.

GastroPy is designed to serve several overlapping user communities.
\emph{Neuroscientists} studying brain-body interactions can use the complete
EGG-fMRI coupling pipeline to generate whole-brain PLV maps and identify
brain regions synchronized with the gastric rhythm.
\emph{Psychophysiologists} conducting standalone EGG studies can use the core
processing pipeline for spectral analysis, rhythm characterization, and
artifact detection without requiring any neuroimaging dependencies.
\emph{Clinical researchers} investigating gastric motility disorders can
leverage the automated quality assessment and multi-band decomposition to
characterize dysrhythmias across the bradygastric, normogastric, and
tachygastric ranges. \emph{Methods developers} building novel analysis
pipelines can use individual functions (e.g., phase extraction, PLV
computation, surrogate generation) as composable building blocks within their
own workflows.

The bundled sample data---three fMRI-EGG sessions from a concurrent recording
study and one standalone EGG recording from the \citet{Wolpert2020} normative
dataset---provides immediate access to realistic data for learning, testing,
and benchmarking, without requiring researchers to acquire or share their own
recordings. Larger preprocessed fMRI BOLD datasets are available via the
\texttt{fetch\_fmri\_bold} function, which downloads data from GitHub Releases
using the Pooch library for reliable, cached retrieval.

\subsection{Reproducibility Considerations}

A core motivation for GastroPy is to improve reproducibility in EGG research.
Currently, many labs use custom analysis scripts that are not publicly
available, may contain undocumented parameter choices, and are difficult to
verify or replicate. By providing a tested, documented, and version-controlled
implementation of standard EGG analysis methods, GastroPy enables researchers
to report their analysis parameters precisely (e.g., by referencing specific
function names and parameter values) and to share complete, executable analysis
pipelines alongside their publications.

All stochastic operations in GastroPy (surrogate generation for PLV testing)
accept explicit random seeds, ensuring bitwise reproducibility. The package
pins minimum dependency versions and is continuously tested on multiple
platforms via GitHub Actions. The automated test suite of 178 tests provides
regression protection against inadvertent changes to algorithm behavior across
releases.

\subsection{Limitations and Future Directions}

Several limitations of the current release should be noted. First, the
coupling module currently supports only the PLV metric; future releases will
add additional coupling measures such as amplitude-phase coupling, coherence,
and information-theoretic metrics, along with group-level statistical testing
for multi-subject studies (e.g., permutation-based cluster correction for PLV
maps). Second, while the core signal processing modules are modality-agnostic,
the neuroimaging layer currently supports only fMRI; planned extensions include
EEG-EGG and MEG-EGG coupling via the \texttt{gastropy.neuro.eeg} and
\texttt{gastropy.neuro.meg} submodules, which will leverage MNE-Python's
existing infrastructure for high-frequency electrophysiology data. Third, the
package does not yet include fully BIDS-compatible data loading, though
preliminary I/O support for BrainVision, EDF, and CSV formats is available in
the \texttt{gastropy.io} module. Fourth, the current artifact detection
approach operates at the cycle level; future work may incorporate continuous
artifact scoring, automated channel interpolation, and integration with
independent component analysis for artifact removal.

We also plan to expand the sample data library with additional recordings
spanning different experimental conditions (fasting vs. postprandial, clinical
populations, pharmacological challenges), develop interactive tutorials as
Jupyter notebooks demonstrating common analysis workflows, and establish
cross-validation benchmarks against existing MATLAB implementations to
quantify numerical agreement. Integration with complementary biosignal
packages such as Systole \citep{Legrand2020} for cardiac-gastric interaction
analysis represents another promising direction for multimodal brain-body
research.

\subsection{Conclusion}

GastroPy fills a critical gap in the scientific Python ecosystem by providing
the first dedicated toolkit for electrogastrography signal processing and
gastric-brain coupling analysis. Its modular architecture, domain-specific
defaults, comprehensive test suite, and bundled sample data make it accessible
to both newcomers to EGG analysis and experienced researchers transitioning
from MATLAB-based workflows. By consolidating validated implementations of
EGG analysis methods into a single, well-documented package, GastroPy aims to
reduce the methodological variability that currently limits cross-study
comparisons and to accelerate the adoption of best practices in gastric-brain
coupling research. As interest in the brain-gut axis continues to grow across
neuroscience, psychiatry, gastroenterology, and human-computer interaction, we
hope that GastroPy will serve as a foundation for reproducible, collaborative
research on gastric-brain interactions and their role in health and disease.

% ============================================================================
% AI USAGE DISCLOSURE
% ============================================================================
\section*{AI Usage Disclosure}

Generative AI tools were used during the development of GastroPy.
Specifically, Anthropic Claude (Claude Code CLI, models
claude-sonnet-4-20250514 and claude-opus-4-20250514) was used for code
generation, test writing, documentation drafting, and debugging across all
modules. All AI-generated code was reviewed, tested, and validated by the
authors. Architectural decisions, algorithm selection, and scientific
methodology were determined by the authors. The automated test suite verifies
correctness of all implementations. The authors accept full responsibility for
the accuracy, originality, and licensing of all code and documentation.

% ============================================================================
% ACKNOWLEDGEMENTS
% ============================================================================
\section*{Acknowledgements}

We acknowledge support from the Lundbeck Foundation (R380-2021-1538,
R140-2013-13057), the European Research Council (ERC-2020-COG, 101001893,
EMBODIED-COMPUTATION), and Aarhus University.

% ============================================================================
% REFERENCES
% ============================================================================
\bibliographystyle{apalike}
\bibliography{paper}

\end{document}
